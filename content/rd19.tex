% Vector variables (in Bold Font style)
\newcommand{\va}{\mathbf{a}} \newcommand{\vb}{\mathbf{b}} \newcommand{\vc}{\mathbf{c}} \newcommand{\vd}{\mathbf{d}} \newcommand{\ve}{\mathbf{e}} \newcommand{\vf}{\mathbf{f}} \newcommand{\vg}{\mathbf{g}} \newcommand{\vh}{\mathbf{h}} \newcommand{\vi}{\mathbf{i}} \newcommand{\vj}{\mathbf{j}} \newcommand{\vk}{\mathbf{k}} \newcommand{\vl}{\mathbf{l}} \newcommand{\vm}{\mathbf{m}} \newcommand{\vn}{\mathbf{n}} \newcommand{\vo}{\mathbf{o}} \newcommand{\vp}{\mathbf{p}} \newcommand{\vq}{\mathbf{q}} \newcommand{\vr}{\mathbf{r}} \newcommand{\vs}{\mathbf{s}} \newcommand{\vt}{\mathbf{t}} \newcommand{\vu}{\mathbf{u}} \newcommand{\vv}{\mathbf{v}} \newcommand{\vw}{\mathbf{w}} \newcommand{\vx}{\mathbf{x}} \newcommand{\vy}{\mathbf{y}} \newcommand{\vz}{\mathbf{z}} 
\newcommand{\vA}{\mathbf{A}} \newcommand{\vB}{\mathbf{B}} \newcommand{\vC}{\mathbf{C}} \newcommand{\vD}{\mathbf{D}} \newcommand{\vE}{\mathbf{E}} \newcommand{\vF}{\mathbf{F}} \newcommand{\vG}{\mathbf{G}} \newcommand{\vH}{\mathbf{H}} \newcommand{\vI}{\mathbf{I}} \newcommand{\vJ}{\mathbf{J}} \newcommand{\vK}{\mathbf{K}} \newcommand{\vL}{\mathbf{L}} \newcommand{\vM}{\mathbf{M}} \newcommand{\vN}{\mathbf{N}} \newcommand{\vO}{\mathbf{O}} \newcommand{\vP}{\mathbf{P}} \newcommand{\vQ}{\mathbf{Q}} \newcommand{\vR}{\mathbf{R}} \newcommand{\vS}{\mathbf{S}} \newcommand{\vT}{\mathbf{T}} \newcommand{\vU}{\mathbf{U}} \newcommand{\vV}{\mathbf{V}} \newcommand{\vW}{\mathbf{W}} \newcommand{\vX}{\mathbf{X}} \newcommand{\vY}{\mathbf{Y}} \newcommand{\vZ}{\mathbf{Z}} 
% Greek Vector variables (in Bold Font style)
\newcommand{\vbeta}{\bm{\beta}} 
\newcommand{\vbhat}{\bm{\hat \beta}}
\newcommand{\vbstar}{\bm{\beta^*}}
\newcommand{\veps}{\bm{\epsilon}}
\newcommand{\vmu}{\bm{\mu}}
\newcommand{\vtheta}{\bm{\theta}}
\newcommand{\valpha}{\bm{\alpha}}
\newcommand{\vdelta}{\bm{\delta}}

% Constant vectors
\newcommand{\vzero}{\mathbf{0}}
\newcommand{\vone}{\mathbf{1}}
\newcommand{\thetac}{\mathrm{\theta^{(0)}}}

% Scalars

% Matrix variables (in DS style where possible)
\newcommand{\mA}{\mathds{A}} \newcommand{\mB}{\mathds{B}} \newcommand{\mC}{\mathds{C}} \newcommand{\mD}{\mathds{D}} \newcommand{\mE}{\mathds{E}} \newcommand{\mF}{\mathds{F}} \newcommand{\mG}{\mathds{G}} \newcommand{\mH}{\mathds{H}} \newcommand{\mI}{\mathds{I}} \newcommand{\mJ}{\mathds{J}} \newcommand{\mK}{\mathds{K}} \newcommand{\mL}{\mathds{L}} \newcommand{\mM}{\mathds{M}} \newcommand{\mN}{\mathds{N}} \newcommand{\mO}{\mathds{O}} \newcommand{\mP}{\mathds{P}} \newcommand{\mQ}{\mathds{Q}} \newcommand{\mR}{\mathds{R}} \newcommand{\mS}{\mathds{S}} \newcommand{\mT}{\mathds{T}} \newcommand{\mU}{\mathds{U}} \newcommand{\mV}{\mathds{V}} \newcommand{\mW}{\mathds{W}} \newcommand{\mX}{\mathds{X}} \newcommand{\mY}{\mathds{Y}} \newcommand{\mZ}{\mathds{Z}}

\newcommand{\cA}{\mathcal{A}} \newcommand{\cB}{\mathcal{B}} \newcommand{\cC}{\mathcal{C}} \newcommand{\cD}{\mathcal{D}} \newcommand{\cE}{\mathcal{E}} \newcommand{\cF}{\mathcal{F}} \newcommand{\cG}{\mathcal{G}} \newcommand{\cH}{\mathcal{H}} \newcommand{\cI}{\mathcal{I}} \newcommand{\cJ}{\mathcal{J}} \newcommand{\cK}{\mathcal{K}} \newcommand{\cL}{\mathcal{L}} \newcommand{\cM}{\mathcal{M}} \newcommand{\cN}{\mathcal{N}} \newcommand{\cO}{\mathcal{O}} \newcommand{\cP}{\mathcal{P}} \newcommand{\cQ}{\mathcal{Q}} \newcommand{\cR}{\mathcal{R}} \newcommand{\cS}{\mathcal{S}} \newcommand{\cT}{\mathcal{T}} \newcommand{\cU}{\mathcal{U}} \newcommand{\cV}{\mathcal{V}} \newcommand{\cW}{\mathcal{W}} \newcommand{\cX}{\mathcal{X}} \newcommand{\cY}{\mathcal{Y}} \newcommand{\cZ}{\mathcal{Z}}

\newcommand{\ewise}[2]{{#1 * #2}} % element-wise product

\newcommand{\EE}{\mathrm{E}} % Expectation

\newcommand{\g}{\,\vert\,} % for conditional probability, division

\newcommand{\set}[1]{\{{#1}\}}

\section {A fully connected FFNN} \label {sec: fcffnn}
Forward propagation
\begin {equation} \begin {split}
& \text {Initialize the input layer } \vZ^1 = \vX, \vA^1 = \vZ^1 \\
& \text {Propagate all activity forward } \vZ^l = \mW^l \vA^{l-1} + \vB^l, \vA^l = f^l(\vZ^l) \\
\end {split} \end {equation}

$\vZ^l = \mW^l \vA^{l-1} + \vB^l$ is equivalent to 
\begin{equation} \label {eq: nnwts}
\begin{bmatrix}
Z_1 \\
\ldots \\
Z_J \\
\end{bmatrix}
= 
\begin{bmatrix}
W_{1,1} & \ldots & W_{1,K} \\
\ldots & \ldots & \ldots \\
W_{J,1} & \ldots & W_{J,K} \\
\end{bmatrix}
\begin{bmatrix}
A_1 \\
\ldots \\
A_K \\
\end{bmatrix} 
+
\begin{bmatrix}
B_1 \\
\ldots \\
B_J \\
\end{bmatrix}
\end{equation}
Where layer $l$ has $J$ nodes and layer $l-1$ has $K$ nodes.

Backward propagation
\begin {equation} \begin {split}
& \text {Assuming that the loss function is } \cL = 0.5 (\vY - \vA^L)^2 \\
& \qquad \text {Where $\vY$ is the known output corresponding to the input $\vX$} \\
& \qquad \text {Calculate the final error } \nabla_a \cL =  \vA^L - \vY \\
& \text {Initialize  the back propagation } \vdelta^L =  \ewise {\nabla_a \cL} {{f^L}^{'} (\vZ^L)} \\
& \text {Backpropagate the error } \vdelta^l =  \ewise {\left[ {\mW^{l+1}}^T \vdelta^{l+1} \right]} {{f^{l}}^{'} (\vZ^l)}, l \in [2, L-1] \\
& \text {Calculate the gradient of weights } \frac {\partial \cL} {\partial \mW^l} = \vdelta^l {\vA^{l-1}}^T, l \in [2, L] \\
& \qquad \text {Equivalently } \frac {\partial \cL} {\partial W_{j,k}^l} = a_k^{l-1} \delta_j^l, l \in [2, L] \\
& \text {Calculate the gradient of bias weights } \frac {\partial \cL} {\partial \vB^l} = \vdelta^l, l \in [2, L] \\
& \qquad \text {Equivalently } \frac {\partial \cL} {\partial B_j^l} = \delta_j^l, l \in [2, L] \\
\end {split} \end {equation}

Update ($\eta$ is a hyperparameter that is not learned by the FFNN)
\begin {equation} \begin {split}
& \text {Update weights } \mW^l = \mW^l - \eta \frac {\partial \cL} {\partial \mW^l}, l \in [2, L] \\
& \text {Update bias weights } \vB^l = \vB^l - \eta \frac {\partial \cL} {\partial \vB^l}, l \in [2, L] \\
\end {split} \end {equation}

\section {14.310}
\subsection {Diff-in-Diff analysis}
The Diff-in-Diff approach is used to estimate the effect of a treatment using natural experiments. Often it is not possible to treat subjects. However a natural experiment may treat a collection of subjects some of whom are in the control group and some in the treatment g
roup. Membership in the control and treatment group is independent of the event and depends on other natually occurring factors.

Definitions:
\begin {enumerate}
\item $Y_{i, e}$ is the response observation for the ith sample, before ($e = 0$) or after the event ($e = 1$)
\item $T_i$ are categorical variables and $T_i \in \set{0, 1}$. $T_i = 1$ when subject is in the treatment group.  $T_i = 0$ when subject is in the control group
\item $E_e = 1$ for the observations that are after an event has happened and $0$ otherwise. The event applies to both the treatment and control groups and is the equivalent of applying the treatment to both the treatment and control groups
\end {enumerate}

Ordinary Least Squares (OLS) equation:
\begin {equation} \begin {split} 
& Y_{i, e} = \beta_0 + \beta_1 T_i + \beta_2 E_e + \beta_3 T_i E_e + \epsilon_{i, e} \\
& \text {Where $\epsilon_{i, e}$ is the unexplained noise} \\
\end {split} \end {equation}

Interpretation:
\begin {enumerate}
\item $\beta_0$: the average value of $Y_{i, e}$ when $T_i = 0, E_e = 0$
\item $\beta_1$: the average increase in $Y_{i, e}$ from $T_i = 0, E_e = 0$ to $T_i = 1, E_e = 0$
\item $\beta_2$: the average increase in $Y_{i, e}$ from $T_i = 0, E_e = 0$ to $T_i = 0, E_e = 1$
\item $\beta_3$: the average causal effect on $Y$ of being treated. This is the diff-in-diff estimate
\item The difference in average outcome in the treatment group before and after treatment is $(\beta_0 + \beta_1 + \beta_2 + \beta_3) - (\beta_0 + \beta_1) = \beta_3 + \beta_2$
\item The difference in average outcome in the control group before and after treatment is $(\beta_0 + \beta_2) - (\beta_0) = \beta_2$
\item Thus, the difference in differences is $(\beta_3 + \beta_2) - (\beta_2) = \beta_3$
\item The diff-in-diff estimate is obtained by removing the effect of the event represented by $E$ on the control group from the effect of the event represented by $E$ on the treatment group
\end {enumerate}

Remarks:
\begin {enumerate}
\item The Diff-in-Diff relies on a "parallel trends" assumption
\item The parallel trends is stated as "in the absence of treatment, the difference between the ‘treatment’ and ‘control’ group is constant over time."
\end {enumerate}

\subsection {ATE and Selection Bias} 
When we observe a group of subjects, some of whom have been treated and some not, we calculate
\begin {equation} \begin {split}
& \EE(Y_i(1) \g  W_i = 1) - \EE(Y_i(0) \g  W_i = 0) \iff \\
& [ \EE(Y_i(1) \g  W_i = 1) -  \EE(Y_i(0) \g  W_i = 1) ] +  \\
& \qquad \EE(Y_i(0) \g  W_i = 1) -  \EE(Y_i(0) \g  W_i = 0) ] \iff  \\
& \text {Average Treatment Effect} + \text { Selection Bias} \\
\end {split} \end {equation}

\subsection {Sample size}
The sample size required to achieve a certain power is given by:
\begin {equation} \begin {split}
& N = \frac {\left(\Phi^{-1} (1 - \beta) + \Phi^{-1} (1 - \alpha/2) \right)^2} {\frac {\tau^2} {\sigma^2} \gamma (1 - \gamma)} \\
& \beta \equiv \text { probability of type II error} \\
& 1 - \beta \equiv \text { power} \\ 
& \alpha \equiv \text { significance level desired for a two-sided test} \\
& \gamma \equiv \text { fraction of sample in treatment group} \\
& \tau \equiv \text { ATE} \\
\end {split} \end {equation}

$N$ may have to be rounded up (if it is fraction) in order to obtain a conservative estimate of sample size.

\subsection {F Stat}
\begin {enumerate}
\item $SSRu := $ Sum of squared residuals of unrestricted model
\item $SSRr := $ Sum of squared residuals of restricted model
\item $r := $ number of restrictions
\item $k := $ number of covariates
\item $n := $ number of samples
\item $F_{stat} := \frac {\frac {SSRr - SSRu} {r}} {\frac {SSRu} {n - (k+1)}}$. When $F_{stat}$ is large, the null hypothesis can be rejected
\item $F_{stat} \sim F(r, n-(k+1))$ where $F(df1, df2)$ is the F-distribution with $(df1, df2)$ degrees of freedom
\end {enumerate}

\subsection {Fixed effects model} \label {r: fem}
G(factor(admin)) creates a dummy variable, one per region, and includes the dummy variable
in the regression
\begin{lstlisting}[language=R]
library("lfe")
model2 <- felm(sex ~ teasown + post + teapost + 
G(factor(admin)), data = qiandata)
summary(model2)
\end{lstlisting}

\subsection {Instrument variable regression}
To use the ivreg() function directly one needs to setup the call as
\begin{lstlisting}[language=R]
ivreg (y ~ exogenous_vars + endogenous_var | 
exogenous_vars + instrument_var)
\end{lstlisting}

\subsection {Fisher exact test in R}
\begin{lstlisting}[language=R]
library(perm)
rm(list = ls())
cough_severity = matrix(c(3, 5, 0, 4, 0, 1), nrow=6, ncol=1, byrow=TRUE)
cough_severity
n_treatment = 3
n_control = 3
n = n_treatment + n_control

permutations = chooseMatrix(n, n_treatment)
permutations
num_permutations = nrow(permutations)

treatment_avg = (1/n_treatment)* permutations %*%cough_severity

control_avg = (1/n_control)*(1-permutations)%*%cough_severity

test_statistic =  abs(treatment_avg-control_avg)

actual_obs_row = 20
observed_test_statistic = test_statistic[actual_obs_row]

by_chance = (test_statistic >= observed_test_statistic)
total_by_chance = sum(by_chance)
\end{lstlisting}
